\documentclass[11pt,a4paper]{hsfuldabook}
\usepackage[ngerman]{babel}
\usepackage[utf8]{inputenc}
\usepackage[T1]{fontenc}
\usepackage{hyperref}

\begin{document}

\title{8085-Simulator}
\subtitle{Benutzerhandbuch}
\author{Florian Schleich}
\publishers{}
\uppertitleback{Entwickelt im Rahmen der Lehrveranstaltung "Mikrocontrollerprogrammierung"
an der Hochschule Fulda.}
\maketitle

\section*{Lizenzvereinbarung}
Die Software steht, ebenso wie dieses Dokument, unter der MIT Lizenz.

\begin{verbatim}
The MIT License (MIT)

Copyright (c) 2014 Florian Schleich (florian.schleich@informatik.hs-fulda.de)

Permission is hereby granted, free of charge, to any person
obtaining a copyof this software and associated documentation
files (the "Software"), to deal in the Software without
restriction, including without limitation the rights to use,
copy, modify, merge, publish, distribute, sublicense,
and/or sell copies of the Software, and to permit persons to
whom the Software is furnished to do so, subject to the
following conditions:

The above copyright notice and this permission notice shall be
included in all copies or substantial portions of the Software.

THE SOFTWARE IS PROVIDED "AS IS", WITHOUT WARRANTY OF ANY KIND,
EXPRESS OR IMPLIED, INCLUDING BUT NOT LIMITED TO THE WARRANTIES
OF MERCHANTABILITY, FITNESS FOR A PARTICULAR PURPOSE AND
NONINFRINGEMENT. IN NO EVENT SHALL THE AUTHORS OR COPYRIGHT
HOLDERS BE LIABLE FOR ANY CLAIM, DAMAGES OR OTHER LIABILITY,
WHETHER IN AN ACTION OF CONTRACT, TORT OR OTHERWISE, ARISING FROM,
OUT OF OR IN CONNECTION WITH THE SOFTWARE OR THE USE OR OTHER
DEALINGS IN THE SOFTWARE.
\end{verbatim}

\tableofcontents

\chapter{Einleitung}
Dieses Dokument soll Ihnen bei der Benutzung des 8085-Simulators helfen. Im Handbuch werden der
Ablauf der Simulation und die Handhabung der Software erläutert.

\subsection{Quelldateien}
Die Software ist Open-Source. Das bedeutet die Quelldateien stehen jedem Interessierten zum Download
zur Verfügung. Die Quelldateien liegen in einem git-Repository auf\\
\url{https://github.com/thetodd/8085-Simulator}.

Ein Anmelden bei GitHub ist nicht notwendig. Das Repository lässt sich auch mit anderen Tools
klonen und selbst erstellen. Eine Anleitung zum Erstellen des Programmcodes aus den Quelltexten befindet
sich im Anhang.

\subsection{Bei Fehlern oder Anregungen}
Sollte Ihnen während der Benutzung dieses 8085-Simulators ein Fehler im Programm auffallen, dürfen
Sie gerne einen Bug im GitHub-Projekt (\url{https://github.com/thetodd/8085-Simulator/issues}) melden,
oder eine E-Mail an florian.schleich@informatik.hs-fulda.de schreiben. Wir sind stets bestrebt die
Software von allen Fehlern und Unschönheiten zu befreien, aber auch uns passiert einmal ein Fehler.
Wenn nette Menschen wie Sie uns Fehler melden, werden wir diese auch beseitigen.

Haben Sie Anregungen, wie die Software noch ein Stück besser werden kann? Dann nur her damit. Eine
E-Mail an die Entwickler genügt. Wenn uns die Idee überzeugt, werden wir versuchen sie umzusetzen.

\chapter{Systemvoraussetzungen}
Derzeit unterstützt die Software nur Microsoft Windows Betriebssysteme. Der Simulator unterstützt
Windows ab XP (SP3). Grundsätzlich werden 32- sowie 64-Bit Betriebssysteme unterstützt.
\begin{table}[h!]
	\begin{tabular}{l || l | l}
	 & Mindestens & Empfohlen \\
	 \hline
	 Betriebssystem & Windows XP (SP3) & Windows 7\\
	 Prozessorleistung & 1 GHz & 1,7 GHz \\
	 Arbeitsspeicher & 512MB & 1GB \\
	 Festplattenplatz & 100 MB & 150MB \\
	 Auflösung & 800x600 & 1280x720 \\
	\end{tabular}
	\caption{Systemvoraussetzungen}
\end{table}

\chapter{Das Programmfenster}
In diesem Kapitel lernen Sie das Programmfentser und die Funktionen der einzelnen Programmteile kennen.

\section{Programmteile}
\subsubsection{Register}
Im oberen Bereich des Programmfensters befindet sich die Übersicht über alle Register des simulierten
Prozessors. Der aktuelle Stackpointer und der Programmzähler werden dort auch angezeigt. Die Werte
können aber dort nicht angepasst werden. Die Textfelder dienen lediglich zur Ansicht, welche Werte
aktuell in den Registern stehen.

\subsubsection{Das Menü}
Oberhalb der Registeranzeige, befindet sich das Menü. Viele Menüelemente lassen sich alternativ auch
über die Tastatur aufrufen. Die zugehörigen Shortcuts befinden sich bei den Menüelementen immer rechts
neben dem Namen.

\subsubsection{Flags}
Rechts neben der Anzeige der einzelnen Registerinhalte befindet sich die Statusanzeige des
Flagregisters. Ein grüner Haken bedeutet in diesem Fall ein gesetztes Flag, ein roter Strich ein
nicht gesetztes Flag.

\subsubsection{Quelltexteditor}

\subsubsection{Programmspeicher}

\subsubsection{Einstellungen}

\subsubsection{Programmübersicht}

\section{Das Menü}
\subsection{Simulationsmenü}

\subsection{Anzeigenmenü}

\chapter {Ein Projekt simulieren}
In diesem Kapitel lernen Sie, wie sie einen Quelltext eingeben, kompilieren und schließlich auf einem
virtuellen Prozessor simulieren können.

\section{Eingeben des Programmtextes}


\end{document}
